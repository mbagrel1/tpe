\documentclass[11pt, a4paper]{article}
\usepackage{tbagrelstandard}
\usepackage{charter}
\geometry{margin=1.5cm}

%\usepackage{caption}
%\usepackage{setspace}
%\usepackage{bookman}
%\usepackage{soul}
%\usepackage{ulem}
%\usepackage{eurosym}
\usepackage{hyperref}
%\usepackage{multirow}
%\usepackage{verbatim}
%\usepackage{moreverb}
%\usepackage{listings}
%\usepackage{amsmath}
%\usepackage{amssymb}
%\usepackage{mathrsfs}
%\usepackage{graphicx}
%\usepackage{titlesec}

\newenvironment{souspar}[1]%
{\begin{list}{}%
         {\setlength{\leftmargin}{#1}}%
         \item[]%
}
{\end{list}}

\begin{document}
\pagestyle{empty}

\heading{\sc{Bagrel}}{Marie}{Année 2017 - 2018}{\sc{ps-1}}{\large}

\fancytitle{Fiche de synth\`{e}se des \sc{tpe}}{0.55\linewidth}{0.25cm}{\LARGE\bfseries}

\bigskip

\paragraph*{}
Poussés par notre intérêt commun pour les nouvelles technologies, Lucie \sc{Chevallereau}, Nicolas \sc{Gobillard} et moi même avons choisi de faire équipe pour réaliser ces travaux personnels encadrés. Nous nous sommes dirigés vers le thème ``L'aléatoire, l'insolite, le prévisible'' qui présentait une polyvalence plutôt attractive. Dans ce thème, de nombreux sujets pouvaient correspondre à nos centres d'intérêts communs mais la ``réalité virtuelle'' était le sujet qui nous attirait le plus en raison de son caractère novateur. Il a fallu, par la suite, préciser le sujet et dégager une problématique. Nous avons donc, pour cela, effectué de nombreuses recherches documentaires au CDI mais aussi sur Internet afin de bien dicerner les enjeux concernant la VR. Nous avons alors trouvé de nombreuses informations sur les bienfaits de la réalité virtuelle en temps que traitement thérapeutique, mais aussi pour les entreprises et pour la simulation de véhicules. Cependant, nous avons également constaté que de nombreux sites internet évoquaient les dangers probables de la réalité  virtuelle, notamment au niveau du cerveau et des yeux. Une question naturelle est alors apparue : ``La réalité virtuelle a-t-elle de réels effets sur l'organisme ?''.
Nous souhaitions, pour nous assurer d'avoir une problématique adéquate, demander l'avis d'un professeur. Nous sommes donc allés consulter M\tss{me} \sc{Dornier}, professeure de SVT au Lycée \sc{Varoquaux}. Elle nous a renseigné sur les études menées actuellement à l'Université de Lorraine pouvant se rapprocher de notre sujet. Elle nous a aussi fortement conseillés d'affiner davantage la problématique car cette dernière semblait encore trop vague.
Après quelques recherches poétiques, nous avons finalement trouvé
\begin{quote}
La réalité virtuelle est-elle une révolution pour l'immersion ou un danger pour la santé ?
\end{quote}

\paragraph*{}
Cette question épouse parfaitement selon nous le sujet que nous voulons traiter, compte-tenu des recherches préalables que nous avons effectuées.

\paragraph*{} Nous avons donc commencé à réaliser notre projet sur la réalité virtuelle. Durant la première séance, nous avons défini les points importants sur lesquels il fallait faire des recherches pour élaborer un plan provisoire. Nous avons choisi en priorité les dangers de la réalité virtuelle mais aussi les nombreux dommines dans lesquels elle peut être utile comme le secteur médical, le marketing\ldots{}
Nous avons distribué le travail pour une meilleure efficacité : Lucie s'est chargée de la définition et de l'étymologie, Nicolas a cherché les principaux casques de réalité virtuelle. Quant à moi, j'ai commencé à rédiger l'historique de la réalité virtuelle. Après deux séances, nous avions achevé ces différentes tâches. Nous avons ensuite défini clairement notre plan en dégageant 5 grandes parties à savoir la définition de la VR (Virtual Reality), le fonctionnement de cette dernière, le fonctionnement de l'\oe{}il et du cerveau et enfin les avantages de cette technologie ainsi que ses dangers potentiels. Nous avons alors, durant les nombreuses séances suivantes, récolté des informations sur internet ainsi que dans nos manuels pour former notre rapport. Lucie s'est concentrée sur les avantages pour le domaine médical, marketing, de l'aéronautique, du shopping mais aussi pour l'armée. Nicolas, quant à lui, s'est chargé de décrypter le fonctionnement d'un casque de réalité virtuelle : l'Occulus rift. Au bout de deux séances, nous nous sommes rendus compte que l'HTC Vive était en réalité le casque le plus intéressant en raison de ses fonctionnalités, d'après les informations à notre disposition. J'ai pour ma part travaillé sur les différences entre la réalité virtuelle et la réalité augmentée, puis je me suis concentrée sur l'oeil et le cerveau. Je me suis bien-sûr aidée d'internet mais aussi de mon manuel de SVT car, en première scientifique, deux chapitres sont dédiés à ces organes ainsi qu'aux différents phénomènes permettant la vision. Nous nous réunissions à la fin de chaque séance pour faire un bilan et une mise en commun de nos travaux. Nous avons également passé une séance en commun pour trouver des professionels à joindre. Nous avons trouvé notamment un médecin spécialiste en thérapie par la réalité virtuelle, qui malheureusement se trouvait trop loin pour que l'on puisse envisager une rencontre.

\paragraph*{} En novembre, nous avons décidé de contacter des professionels pour tester la réalité virtuelle, afin d'avoir une expérience sur le sujet. Nous nous sommes rendus à la porte ouverte d'\it{Epitech}, où nous avons pu découvrir l'école mais aussi recueillir l'avis d'étudiants sur la réalité virtuelle mais aussi sur la réalité augmentée, mais nous n'avons malheureusement pas pu tester un casque de réalité virtuelle à cause de travaux à cette période. \`{A} la suite de cette visite, nous avons étoffé notre dossier, plus précisement la première partie où nous différencions les différents types de réalités. Au milieu du mois de décembre, nous avons pris contact avec la société \it{Human Games} en espérant pouvoir tester la réalité virtuelle. Nous nous sommes donc rendus le 28 décembre dans leurs locaux. Nous avons donc été reçu de façon très chaleureuse et nous avons pu recueillir des précisions sur les différents types de réalités et surtout sur les différences entre les différents casques VR. Dans la seconde partie de notre visite, nous avons eu la chance de tester l'HTC Vive dans un environnement virtuel avec différentes activités proposées. Les sensations étaient au rendez-vous et nous n'avons pas ressenti d'effets néfastes lors de l'utilisation du dispositif, mais seulement quelques minutes après, au moment d'enlever le casque. Grâce à ce précieux entretien, nous avons pu ajouter dans notre dossier de nombreuses informations tirée d'une expérience réelle.

\paragraph*{} Durant les 3 séances suivantes, nous avons mis au propre le contenu des premières parties et j'ai approfondi certains points scientifiques comme l'explication des vertiges en raison du fonctionnement de l'oreille interne ou les dangers de la lumière bleue. Au milieu du mois de janvier, nous avons décidé de remettre en forme la totalité de ce TPE. Initialement, le rapport était rédigé à l'aide d'un traitement de texte conventionnel, hors nous avions de nombreux problèmes de mise en page, notamment avec les photographies car chacun ne disposait pas du même logiciel. J'ai donc choisi de le mettre en page avec \LaTeX{}, qui est utilisé pour mettre en page des pubications scientifiques. Il m'a fallu environ douze heures hors du temps scolaire pour rédiger totalement notre production. J'ai alors sollicité les membres de ma famille pour le relire et le corriger.

\paragraph*{}
Nous sommes donc arrivé à la conclusion que la réalité virtuelle n'était qu'au début de son développement mais qu'elle permettait déjà des avancées ou révolutions dans de nombreux domaines. Nous avons, contrairement à notre pensée initiale, remarqué que les effets néfastes supposés de la réalité virtuelle n'avaient pas encore été prouvés par les scientifiques en raison du manque de recul sur cette nouvelle technologie. Un comportement raisonnable est donc à adopter avec les nouveaux dispositifs de réalité virtuelle.

Pour mener à bien notre projet, il a fallu, pour ma part, que je mobilise beaucoup de temps pour reformuler et mettre en page le rapport final, toute seule. Nous aurions pu mieux nous organiser pour la mise en page, et prévoir plus tôt la convertion en \LaTeX{}. Nous aurions également pu centrer davantage notre problématique sur la réalité virtuelle pour les jeux vidéos, avec par exemple une étude auprès des joueurs. Cependant, nous avons, selon moi, réussi à former un trio sérieux grâce auquel nous avons pu apprendre de nombreuses choses sur la réalité virtuelle ainsi que sur notre organisme.

\paragraph{Remerciements}
Nous remercions M. \sc{Julian}, professeur de mathématiques qui nous a guidé tout au long de notre projet et qui nous a apporté de précieux conseils. \\
Nous remercions également l'école \it{Epitech} qui a accepté de répondre à nos questions mais qui nous a également permis de visiter ses locaux.
Nous remercions tout particuièrement l'entreprise \it{Human Games} pour nous avoir si bien accueilli et pour nous avoir permis de réaliser une véritable expérience de réalité virtuelle.
Merci à nos professeurs pour leurs conseils pour ce projet et merci aux élèves du lycée d'avoir répondu à notre sondage.
Enfin, je remercie mes proches, tout particulièrement mon frère, pour la relecture et l'aide à la mise en page.

\paragraph{Bibliographie}

\begin{itemize}
  \setlength{\itemsep}{-0.3em}
  \item \href{https://www.realitevirtuelle360.com/definition-qu-est-ce-que-la-realite-virtuelle/}{\tt{www.realitevirtuelle360.fr}}
  \item \href{http://www.lemonde.fr/grands-formats/visuel/2017/04/17/comment-la-realite-virtuelle-joue-avec-votre-cerveau\_5112541\_4497053.html}{\tt{www.lemonde.fr}}
  \item \href{https://gizmodo.com/5994263/whats-inside-the-oculus-rift-virtual-reality-headset}{\tt{www.gizmodo.com}}
  \item \href{http://0540044e.esidoc.fr/search.php?pid=\&action=Record\&id=MF\_MF1501211431968\_2015\&num=3\&total=14}{\tt{www.esidoc.fr}}
  \item \href{https://fr.wikipedia.org/wiki/R%C3%A9alit%C3%A9_virtuelle}{\tt{www.wikipedia.org}}
\end{itemize}

\vfill
\begin{center}
\rule{0.90\linewidth}{0.5pt}\\
\small
- Marie \sc{Bagrel} - Première Scientifique 1 - Année 2017-2018 - Lycée Arthur \sc{Varoquaux} -
\end{center}

\end{document}